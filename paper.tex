\documentclass[journal, a4paper]{IEEEtran}
\usepackage{graphicx}   % Written by David Carlisle and Sebastian Rahtz
\usepackage{url}        % Written by Donald Arseneau
\usepackage{amsmath}    % From the American Mathematical Society
\usepackage{hyperref}
\usepackage[backend=biber,style=numeric,sorting=none]{biblatex}
\addbibresource{refs.bib}
% Your document starts here!
\begin{document}

% Define document title and author
	\title{Acoustic Resonance}
	\author{Gabriel Massie
	\thanks{Advisor:~Spicklemire, Stephen J.}}
	\markboth{Senior Research, report 1}{}
	\maketitle

% Write abstract here
\begin{abstract}
(\textit{Use this template for your report.}) Replace this section with your abstract. An abstract is a short (50--80 words, roughly) paragraph intended to give the reader a {\it brief} overview of the work and the relevant results. The abstract is what the reader will use to decide whether it's worth their time to read your paper. Make the abstract short, but interesting! You want to draw the reader in. If they don't read your paper, then it will have been a waste of your time.
\end{abstract}

% Each section begins with a \section{title} command
\section{Introduction}
	% \IEEEPARstart{}{} creates a tall first letter for this first paragraph
	\IEEEPARstart{T}{his} section describes the physical context of the investigation, and research question to be answered and some insight into the motivation for the investigation. Like the abstract, the introduction should provide some incentive for them to \textit{keep} reading. Why is this work interesting? Why is it important? What novel approach did you take? You might hint at the fascinating results you produced, or the creative way you dealt with experimental challenges.
% Theory and Background
\section{Theory and Background}
	% LaTeX takes complete care of your document layout ...
	This is where you provide references to background information or theory that the reader might need to understand your project. This could include textbook references or other resources. Anything you don't work out here, but that you need for to complete the analysis, should be referenced.

\section{Experimental apparatus and set up, Data Acquisition, and Collected Data}
	The setup was a speaker using digilent waveforms software, swept frequencies of 50Hz-1KHz pointed at a resonator with one closed end and later both ends closed. Data collection was done with a microphone. Data analysis of the wav files generated by audacity

\section{Analysis}
	How did you analyze the data to answer the research question? Explain. This should be clear and complete enough that the reader could reproduce your results. You may link to notebooks that do the actual analysis. Describe how you estimated the uncertainty in the results. This can often be the most extensive section in the paper because the quality and value of the results depends on the care with which you perform the analysis. Give the reader reasons to feel confidence in the correctness of your results, and the ensuing assertions you make based on those results.

\section{Discussion}
	What did you learn from the data? How was the result significant? How does the result relate to the original research question? What future improvements can you suggest, and why? These are the kinds of questions answered in this section. This is where you make your final claims about the results you produced, where you pull everything together. This is the grand finale!
    
\section{Format}
	The goal for this course is not only for you to engage in more in-depth experimental work, but also to prepare you to create publication quality reports using the same tools used by professionals in the field. This means you'll be learning some new tools!

	The primary new tool is the \LaTeX{} system. The good news is that you don't have to figure out how to \textit{install} \LaTeX{} yourself, since you're already using a GitHub codespace to run latex in a browser. Your reports will be written in \LaTeX{} using this template. I recognize that this may be challenging, especially in the beginning, but I'm happy to help! It will get easier.
	
	% You can cite a book or paper by using \cite{reference}.
	% The references will be defined at the end of this .tex file in the bibliography

	References should be saved in the {\tt refs.bib} file and should be cited by name (example: ``... as shown by Einstein, 1905 \parencite{einstein}, ...''). The BibLaTeX system will automatically create the references list based on which references are cited in the text \parencite{dirac}.

	References should be of academic character and should be published and accessible. Your advisor can answer your questions regarding literature research. You must cite all used sources.	Examples of good references are text books and scientific journals or conference proceedings.	If possible, citing internet pages should be avoided. In particular, Wikipedia is \emph{not} an appropriate reference in academic reports. Avoiding references in languages other than English is recommended.

	% You can reference tables and figure by using the \ref{label} command. Each table and figure needs to have a UNIQUE label.

	Figures and tables should be labeled and numbered, such as in Table~\ref{tab:simParameters} and Fig.~\ref{fig:tf_plot}. You may include raw URLs using the \texttt{\textbackslash url} \LaTeX{} command, like this: \url{http://www.google.com}, or if you want to create a link using other text you can do that \href{https://www.google.com}{this way.}

	% This is how you define a table: the [h!bt] means that LaTeX is forced (by the !) to place the table exactly here (by h), or if that doesnt work because of a pagebreak or something, it tries to place the table to the bottom of the page (by b) or the top (by t).

	\begin{table}[h!bt]
		% Center the table
		\begin{center}
		% Title of the table
		\caption{Simulation Parameters}
		\label{tab:simParameters}
		% Table itself: here we have two columns which are centered and have lines to the left, right and in the middle: |c|c|
		\begin{tabular}{|c|c|}
			% To create a horizontal line, type \hline
			\hline
			% To end a column type &
			% For a linebreak type \\
			Information message length & $k=16000$ bit \\
			\hline
			Radio segment size & $b=160$ bit \\
			\hline
			Rate of component codes & $R_{cc}=1/3$\\
			\hline
			Polynomial of component encoders & $[1 , 33/37 , 25/37]_8$\\
			\hline
		\end{tabular}
		\end{center}
	\end{table}

If you have questions about how to write mathematical formulas in LaTeX, please read a LaTeX book or the \href{https://tobi.oetiker.ch/lshort/lshort.pdf}{Not So Short Introduction to LaTeX}

% This is how you include a pdf figure in your document. 
	\begin{figure}[h!bt]
		% Center the figure.
		\begin{center}
		% Include the pdf file, scale it such that it's width equals the column width. You can also put width=8cm for example...
		\includegraphics[width=\columnwidth]{figs/pendulum.pdf}
		% Create a subtitle for the figure.
		\caption{Comparison of experimental results and theoretical prediction.}
		% Define the label of the figure. It's good to use 'fig:title', so you know that the label belongs to a figure.
		\label{fig:tf_plot}
		\end{center}
	\end{figure}

Here is another figure, Fig.~\ref{fig:tf_plot2} produced by the notebook in the same directory as this file.

\begin{figure}[!hbt]
	\centering
	\includegraphics[width=\columnwidth]{figs/sine_cosine_plot.png}
	\caption{Some random plots as an example}
	\label{fig:tf_plot2}
\end{figure}

% Now we need a bibliography:
\printbibliography[title={References}]

% Your document ends here!
\end{document}
